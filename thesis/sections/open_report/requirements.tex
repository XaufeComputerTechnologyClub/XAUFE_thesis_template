\begin{center}
  \zihao{-2} \heiti
  \textbf{填写说明}
\end{center}

\vspace{0.8cm}
% \renewcommand{\baselinestretch}{1.5}
{\setstretch{2}
1.开题报告应在指导教师的指导下,
由学生在毕业论文(设计)工作前期内完成,
经指导教师签署意见,所在系(所、学部)和学院审核后生效。

2.开题报告内容必须用黑墨水笔工整书写或按教务处统一设计的电子文档标准格式(可从教务处网址上下载)打印,
禁止打印在其它纸上后剪贴。

3.开题报告的内容要求:

(1)选题背景和意义。
学生应对论题、选题的出发点、相关背景情况、理论和现实需求、研究成果可能具有的学术意义和应用价值做出简要分析、说明。

(2)研究基础和主要参考文献。
学生应对文献资料的收集整理准备情况、参与学术研究和社会调查等情况、已发表论文或已完成相关研究情况等做出说明。

(3)主要研究内容。
学生应对所研究问题的研究范围、学术渊源、国内外已有研究成果和研究动态、研究要点、可能涉及的相关领域和问题、拟采用的基本理论、研究方法及其对本论题的适用情况、论文主体框架等做出明确说明,
对于课题直接相关的已有成果的基本情况,
特别是对已有成果存在的不足和研究空间,
做出分析和判断,
对可能达到的学术目标做出预测。

(4)拟采取的研究方法和技术路线。
依据论题确定具体的研究方法和研究思路。

(5)研究计划及进度安排。
学生应根据自己所确定的论题制订比较详细的研究计划和工作安排。

4.本报告由学生所在学院保存。

5. 若有关内容所留空间不够,可另加附页。
% \renewcommand{\baselinestretch}{1}
}
\newpage
