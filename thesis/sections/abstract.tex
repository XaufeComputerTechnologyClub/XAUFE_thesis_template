\documentclass[../main.tex]{subfiles}
\graphicspath{{figures/}{../figures/}}

\begin{document}
\officialDemand{
  \textbf{中英文摘要和关键词}
  \begin{officialDemandText}
    摘要内容应概括地反映出论文的主要内容,
    说明论文的研究目的、内容、方法、成果和结论。
    语言力求精练、准确,摘要字数300字左右。
    关键词是表述论文主题内容信息的单词或术语,
    是供检索用的主题词条,
    关键词一般3—5个。
    摘要与关键词应在同一页。
    英文摘要和关键词应与中文摘要及关键词的内容相一致,
    译文要准确、规范、流畅。
  \end{officialDemandText}
  \begin{officialDemandText}
      中英文内容摘要及关键词
      
      另起一页。
      论文标题用三号黑体加粗居中,上下各空一行;
      论文(设计)题目下居中打印“摘要”二字,
      用三号黑体,字间空一格,单倍行距。
      “摘要”二字下空一行打印内容,用小四号宋体。
      中文摘要内容下空一行左起打印中文关键词。
      中文的“关键词”三字用小四号宋体加粗,
      内容用小四号宋体,
      每一关键词之间用分号分开,
      最后一个关键词后不打标点符号。
      英文内容摘要及关键词。
      另起一页。

      英文摘要和关键词按照中文摘要和关键词完全翻译。
      英文题目题用三号Times New Roman字体并加粗居中,
      上下各空一行;英文题目下打印英文“Abstract”,
      用三号Times New Roman字体。
      “Abstract”下空一行打印内容,
      用小四号Times New Roman字体。
      英文摘要内容下空一行左起打印英文关键词,
      英文的“Key Words”用小四号Times New Roman字体并加粗,内容用小四号Times New Roman字体,
      每一关键词之间用逗号分开,
      最后一个关键词后不打标点符号。
      英文内容摘要的标点符号用英文形式。
  \end{officialDemandText}
  在这个模板中要使用\lstinline{\\zhEn{中文}{英文}}
  一句中文一句英文进行写作,
  如果不喜欢这样的模式,
  可以将\lstinline{\\PrintChineseAbstract}
  替换为中文摘要,
  将\lstinline{\\PrintenglishAbstract}
  替换为英文摘要内容。
}

\zhEn{%
  推荐用这个命令同时写中英文摘要.%
}{%
  It is recommended to write both Chinese and English abstracts with this command.
}
\zhEn{%
  这样没有翻译的摘要会自动创建TODO.%
}{%
  This does not translate the summary to automatically create Todo item.%
}
\zhEn{例如这是等待翻译的一句话}{}
\ifdraft{
  \newpage
}{}

%
\begin{center}
  \section*{\MyTitle}
\end{center}

\selectlanguage{pinyin}
\begin{abstract}
  \PrintChineseAbstract

	\keywords{MSP430G2553 \qquad 小车 \qquad 小车控制}
\end{abstract}
% \newpage
\selectlanguage{english}
\begin{center}
  \section*{\MyEnglishTitle}
\end{center}
\begin{abstract}
	\PrintenglishAbstract
	\keywords{MSP430G2553, \qquad obstacle car, \qquad control car}
\end{abstract}
\selectlanguage{pinyin}

\end{document}
