\documentclass[../main.tex]{subfiles}
\graphicspath{{figures/}{../figures/}}

\begin{document}
\officialDemand{
  这是毕业论文相关的提示框,
  只在草稿状态下显示。

	\textbf{题目}
	\begin{officialDemandText}
		论文题目要体现出专业特点,
		在专业范围内选题。
		题目应能概括整个论文最重要的内容,
		简短、明确、有概括性;
		字数要适当,一般不宜超过24个汉字,
		必要时可增加副标题。
		外文题目一般不宜超过12个实词。
	\end{officialDemandText}
	本模板使用\lstinline{\\title{}}命令代表题目

  \textbf{格式要求:版式与用字}
  \begin{officialDemandText}
    1.打印用纸及格式要求:
    毕业论文必须统一使用A4纸及WORD格式正反双面打印。

    2.排版要求:毕业论文要求纵向打印,
    论文的文字图形一律从左至右横写横排;
    正文采用小四号宋体字打印。

    3.页面设置:页边距的要求为:
    上(T):2.54cm;下(B):2.54cm;左(L):3cm;
    右(R):2.2cm;行距为固定值20磅。

    4.毕业论文文字、标点要求:字迹必须清楚,
    忌用异体字、复合字及一切不规范的简化字。
    标点符号应按照国家新闻出版署公布的“标点符号使用方法”的统一规定正确使用,
    切忌误用和含糊混乱。

    5.文中的数字,
    除部分结构层次序数和词、词组、惯用语、缩略语、
    具有修辞色彩语句中作为词素的数字必须使用汉字外,
    应当使用阿拉伯数码。
    同一文中,数字表示方法应前后一致。

    6.注意文中代表变量的英文字母必须用斜体,其它用正体。微分号d、圆周率π、自然底数e、矩阵转置T均应为正体。
  \end{officialDemandText}

  \textbf{论文封面格式要求}
  \begin{officialDemandText}
    封面采用学校教务处规定的统一格式,
    必须正确无误。

    (1)论文标题

    论文标题用三号黑体加粗。

    (2)论文副标题

    论文副标题用四号黑体,正标题下居中,文字前加破折号。

    (3)其余项目

    学生姓名、指导教师、班级、专业、学号用三号宋体。
  \end{officialDemandText}

  \textbf{装订要求}
  \begin{officialDemandText}
    1.装订顺序为:
    封面,独创性及知识产权声明(见附件),
    中文摘要及关键词,英文摘要及关键词,目录,正文,
    参考文献,附录(如果有),致谢。

    2.毕业论文(设计)采用左侧装订。
    毕业论文的封面和封底用纸采用特殊A3规格180g/m2皮纹纸包皮装订,
    封皮颜色各二级学院统一要求。
    用于归档的毕业论文(设计)定稿一式两份。
  \end{officialDemandText}
}


\end{document}
