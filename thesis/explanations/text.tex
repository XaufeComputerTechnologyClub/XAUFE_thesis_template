\documentclass[../main.tex]{subfiles}
\graphicspath{{figures/}{../figures/}}

\begin{document}

\officialDemand{
	\textbf{正文}
	\begin{officialDemandText}
		正文是毕业论文的主体和核心部分,
		不同学科专业和不同形式的毕业论文可以有不同的写作方式。
		应当层次分明、数据可靠、文字简炼、说理透彻、推理严谨、立论正确。
		正文一般包括绪论(或序言)、正文主体和结论三部分。
		绪论部分主要论述论文的选题背景、国内外研究现状分析、研究目的和主要内容等。
		正文主体是对研究工作的详细表述,一般由标题、文字、图、表格和公式等部分组成。
		结论应概括说明所进行工作的情况和价值,
		分析其优点和特色,指出创新所在,
		并应指出其中存在的问题和今后的改进方向,
		结论要简单、明确,篇幅不宜过长。
		毕业论文(设计)的篇幅一般不少于8000字。
	\end{officialDemandText}
	\textbf{正文格式要求}
	\begin{officialDemandText}
		正文从奇数页起排,小四号宋体,行距固定值20磅。

		论文应分层次。
		正文中标题分为一级标题、二级标题、三级标题,
		编号统一为:1、1.1、1.1.1。

		四级以后的标题和编号的编排原则为:下级标题的显目程度不超过上一级,
		不重复或混淆。
		如可采用1)、(1)、a、a)、(a)等格式。
		格式:
		\begin{itemize}
			\item 一级标题:1 (左顶格,黑体,小三号。2.5倍行距)
			\item 二级标题:1.1 (左顶格,黑体,四号。2倍行距)
			\item 三级标题:1.1.1 (左顶格,宋体加粗,小四号。1.5倍行距)
			\item 四级以后的标题应用常规小四号宋体。
		\end{itemize}

		论文中不宜用[1][2]和\textcircled{1},\textcircled{2}等作为正文中内容的序号,以免与注释号混淆。
	\end{officialDemandText}
}

\officialDemand{
	\textbf{1)论文中的图示与应用}
	\begin{officialDemandText}

		应当具有以下五个方面的学术提示:
		有引图用语,诸如“×××见图1所示”;有图序编号,
		诸如“图1”;
		有准确的图题标示,即×××××图;
		图的图例标识应说明清楚,比例准确、排列美观;
		文中有用图交待,如“从图1可见……”。

		论文图示应用的具体要求:
		\begin{enumerate}
			\item[a] 插图在全文内按顺序编号,
			如毕业论文第一幅图为图1;
			图题应在图的下方用五号黑体居中排列,
			其英文字体为五号Times New Roman字体。
			图与上文应留一行空格。图名与下文留一空行。

			\item[b] 图形符号及各种线型的画法应符合相关国家标准。
			图中字体均为五号,
			其中中文字符选用宋体,
			数字、字母选用Times New Roman字体。
			坐标图中,必须给出横、纵坐标的“量符号/单位符号”。

			\item[c] 图应有图题,并与图序一起在图片的下方用5号黑体居中排列。

			\item[d] 应遵循“先文后图”、“图不跨节”的规定,
			即在正文叙述中先见引图用语,后见图,
			如第3部分中出现的图,
			不要画到第4部分的文字叙述中去。图片与上下文应留一行空格。

			\item[e] 使用他人的图示应注明出处。
		\end{enumerate}
	\end{officialDemandText}
}

\officialDemand{
	\textbf{2)论文中表格的运用}
	\begin{officialDemandText}

		本科毕业论文(设计)中凡是出现表格时,
		应当具有以下方面的学术提示:有引表用语,
		如“×××详见表2”;表序编号,
		如表2;
		有准确的表名标示,如××××××表;
		表的画法规范统一,表内项目排列规范;
		表的资料交待清楚;
		有用表交待,如“由表2可见”。本科毕业论文(设计)用表的具体要求如下:

		\begin{enumerate}
			\item[a] 表按其在文中顺序编号,如文中第五个表的表序为表5。

			\item[b] 表应有表题,并与表序一起在表的上方用5号黑体居中排列。

			\item[c] 为了使表的格式统一,
				本科毕业论文(设计)一律采用三线表,
				即有三条线,第1和第3条线为1.5磅,
				第2条线为0.25磅。

			\item[d]
			表的各栏均应标明量或测试项目标准规定的符号、单位,
			其表示方法与图中相同,
			特别指出的是,
			当表中所有栏目中单位都相同时,
			应将单位标注在表的右上方,不用“单位”二字。

			\item[e] 先文后表,表不跨节,
			即先见引表用语,表在引表用语之后,
			用表交待在表后。表格与上下文应留一行空格。

			\item[f] 为使表格简洁,
			对表中的符号、标记、代码及需要说明的事项,
			可以用简练的文字以表注的形式用小5号宋体附注于表下,
			不出现“附注”或“注”字样。
			若对整个表的说明,说明文字前用“说明”字样。
			脚注或说明,各项可另起行,也可以接排。

			\item[g] 引用他人的表格须注明出处。
		\end{enumerate}
	\end{officialDemandText}
}

\officialDemand{
	\textbf{3)论文中的公式与应用}
	\begin{officialDemandText}
		公式书写应在文中另起一行,居中书写。
		公式的编号加圆括号,放在公式右边行末,
		公式和编号之间不加虚线。
		公式后应注明编号,该编号按在全文中顺序编排,
		格式为(1),表示全文的第一个公式。

		公式中各物理量及单位采用中华人民共和国国家标准
		GB3100-31002-93《量和单位》,
		禁止使用已废弃的符号和计量单位。

		公式中的字体字号要与正文中的一致,
		不得任意加大加黑公式中的符号和数字,
		也不得随意加大行间距。
	\end{officialDemandText}
}

\officialDemand{
	\textbf{注释}
	\begin{officialDemandText}
		注释是对论文中所应用的名词术语的解释,
		或是对引文出处的说明,采用脚注形式。
	\end{officialDemandText}
	\begin{officialDemandText}
		正文中的加注采用右上角标注,形式为“\textcircled{1}”,
		同时在本页留出适当行数,
		用横线与正文分开,
		空两格后写出相应的注号,
		再写注文。
		注号按
		\textcircled{1}\textcircled{2}
		\textcircled{3}\textcircled{4}排序,
		每个注文各占一段,用小五宋体。
	\end{officialDemandText}
}


\end{document}
